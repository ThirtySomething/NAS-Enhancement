\documentclass[12pt,a4paper]{article}
\usepackage{afterpage}
\usepackage[toc,page]{appendix}
\usepackage{amsmath,amssymb}
\usepackage{array}
\usepackage[ngerman]{babel}
\usepackage[sfdefault,condensed]{cabin}
\usepackage[margin=10pt,font=small,labelfont=bf,labelsep=endash]{caption}
\usepackage{color}
\usepackage{etoolbox}
\usepackage{fancyhdr}
\usepackage{float}
\usepackage{fncychap}
\usepackage[T1]{fontenc}
% \usepackage[showframe]{geometry}
\usepackage{graphicx}
\usepackage{hhline}
\usepackage{imakeidx}
\usepackage[utf8]{inputenc}
\usepackage{latexsym}
\usepackage{listings}
\usepackage{longtable}
\usepackage{marvosym}
\usepackage{microtype}
\usepackage[parfill]{parskip}
\usepackage{pdflscape}
\usepackage{setspace}
\usepackage{stmaryrd}
\usepackage{tabularx}
\usepackage{tocloft}
\usepackage{wasysym}

\fancypagestyle{lscape}{
\fancyhf{}
\fancyfoot[LE]{
\begin{textblock}{20} (1,5){\rotatebox{90}{\leftmark}}\end{textblock}
\begin{textblock}{1} (13,10.5){\rotatebox{90}{\thepage}}\end{textblock}}
\fancyfoot[LO] {
\begin{textblock}{1} (13,10.5){\rotatebox{90}{\thepage}}\end{textblock}
\begin{textblock}{20} (1,13.25){\rotatebox{90}{\rightmark}}\end{textblock}}
\renewcommand{\headrulewidth}{0pt}
\renewcommand{\footrulewidth}{0pt}}

\pagestyle{fancy}
\renewcommand{\cftpartleader}{\cftdotfill{\cftdotsep}}
\renewcommand{\cftsecleader}{\cftdotfill{\cftdotsep}}
\setlength{\headheight}{15pt}

\makeatletter
\def\verbatim{
    \scriptsize,
    \@verbatim,
    \frenchspacing,
    \@vobeyspaces,
    \@xverbatim,
}
\makeatother

\makeatletter
\preto{\@verbatim}{\topsep=0pt \partopsep=0pt }
\makeatother

\author{Jochen Paul}
\title{Ein Server vor einem NAS}
\date{\today}

\usepackage[
pdftex,
pdfauthor={Jochen Paul -- https://www.derpaul.net},
pdftitle={Ein Server vor einem NAS},
pdfsubject={Verlagerung von Funktionalität vom NAS auf einen eigenen Server},
colorlinks=true,
linkcolor=blue,
urlcolor=blue
]{hyperref}
\usepackage[all]{hypcap}

\makeatletter
\newcommand{\mytitle}{\@title}
\makeatother

\pagestyle{fancy}
\fancyhf{}
\renewcommand{\sectionmark}[1]{\markright{#1}}
\renewcommand{\subsectionmark}[1]{\markright{#1}}
\renewcommand{\subsubsectionmark}[1]{\markright{#1}}
\fancyhead[R]{Ein Server vor einem NAS}
\fancyhead[L]{\nouppercase{\rightmark}}
\fancyfoot[C]{Seite \thepage}
\renewcommand{\headrulewidth}{0.4pt}
\renewcommand{\footrulewidth}{0.4pt}

\newcommand{\code}[1]{\textit{#1}}
\newcommand{\jpaimg}[2]{\begin{figure}[H]\centering\fbox{\includegraphics[width=380px]{#1}}\caption{#2}\label{fig:#2}\end{figure}}
\newcommand{\jpaquote}[1]{\glqq{}#1\grqq{}}

\definecolor{dkgreen}{rgb}{0,0.6,0}
\definecolor{gray}{rgb}{0.5,0.5,0.5}
\definecolor{mauve}{rgb}{0.58,0,0.82}

\lstset{frame=tb,
    language=sh,
    aboveskip=3mm,
    belowskip=3mm,
    showstringspaces=false,
    columns=flexible,
    basicstyle={\scriptsize\ttfamily},
    numbers=none,
    numberstyle=\tiny\color{gray},
    keywordstyle=\color{blue},
    commentstyle=\color{dkgreen},
    stringstyle=\color{mauve},
    breaklines=true,
    breakatwhitespace=true,
    tabsize=3
}

\makeindex

\begin{document}
\clearpage\maketitle
\thispagestyle{empty}
\newpage

\thispagestyle{fancy}
\newpage
\clearpage
\tableofcontents
\addtocontents{toc}{\protect\thispagestyle{fancy}}
\newpage

\section{Motivation}
Das \href{https://de.wikipedia.org/wiki/Network_Attached_Storage}{NAS}
\href{https://www.synology.com/en-global/company/news/article/Synology_Unveils_DiskStation_DS411slim}{DS411Slim} von
Synology bietet zahlreiche Features. Unter anderem kann die Funktionalität als reines NAS erweitert werden, zum Beispiel
mit

\begin{itemize}
    \item Einer \href{https://de.wikipedia.org/wiki/MariaDB}{Maria DB}
    \item Einem \href{https://de.wikipedia.org/wiki/MQTT}{MQTT} Broker
    \item Einem \href{https://de.wikipedia.org/wiki/Subversion}{Subversion} Server
    \item Einem \href{https://de.wikipedia.org/wiki/Antivirenprogramm}{Virenscanner}
\end{itemize}

Diese Erweiterungen werden von mir eingesetzt, und zwar zusätzlich zu den ab Werk installierten Features wie

\begin{itemize}
    \item Einem \href{https://de.wikipedia.org/wiki/Digital_Living_Network_Alliance}{DLNA} Server
    \item Einem \href{https://de.wikipedia.org/wiki/Dateiserver}{Dateiserver}
\end{itemize}

Mittlerweile ist das NAS etwas in die Jahre gekommen -- und wird leider nicht mehr von Synology unterstützt
\footnote{\href{https://www.synology.com/de-de/releaseNote/DS411slim}{Siehe Version 6.2.1-23824}}. Es gibt nur noch
Sicherheitsupdates, neue Features oder gar eine neue Variante des Diskstation Manager (DSM), dem Betriebssystem von
Synology für ihre NAS-Produkte, sind ausgeschlossen.

Zudem ist die Hardware nicht leistungsfähig genug, um aktuelle Bedürfnisse zu befriedigen. Der
\href{https://de.wikipedia.org/wiki/Prozessor}{Prozessor} ist laut
\href{https://global.download.synology.com/download/Document/DataSheet/DiskStation/11-year/DS411slim/Synology_DS411slim_Data_Sheet_enu.pdf}{Dokumentation}
ein \href{https://wikidevi.com/wiki/Marvell#Kirkwood}{Marvel Kirkwood}, welcher eine
\href{https://de.wikipedia.org/wiki/ARM-Architektur}{ARMv5} Architektur hat. Dazu kommt, dass das NAS lediglich 256 MB
RAM hat.

Deswegen möchte ich meine \href{https://www.zotac.com/us/product/mini_pcs/id41-plus}{ZBox ID 41+} als Server vor dem
NAS verwenden. Die ZBox hat einen \href{https://de.wikipedia.org/wiki/Intel_Atom}{Intel Atom D525} Prozessor. Diesem
stehen zudem 4 GB RAM zur Seite -- das sollte mehr Performance bieten, als es dem NAS jemals möglich wäre.

\section{Aktueller Zustand}
Aktuell ist ausschließlich das NAS in Verwendung. Da dabei ziemlich viele Dienste darauf laufen, kommt die Hardware
ziemlich schnell an ihre Grenzen. Grob skizziert sieht das in etwa so aus:

\jpaimg{./images/DS411Slim.png}{Aktueller Zustand}

\section{Die Idee}
Die Idee ist, auf der ZBox ein Host-OS einzurichten, und zwar für
\href{https://de.wikipedia.org/wiki/Docker_(Software)}{Docker}. Die oben erwähnten Applikationen sollen dann in Docker
Containern laufen. Als Datenspeicher dient nach wie vor das NAS, jedoch wird der Festplattenplatz des NAS irgendwie
in das Host-OS eingebunden und die Docker Container legen darauf ihre Daten ab. Irgendwie deswegen, da es ja
verschiedene Möglichkeiten gibt:

\begin{itemize}
    \item Als Filesystem
    \begin{itemize}
        \item \href{https://de.wikipedia.org/wiki/Server_Message_Block}{SMB Share}
        \item \href{https://de.wikipedia.org/wiki/Network_File_System}{NFS Share}
    \end{itemize}
    \item \href{https://de.wikipedia.org/wiki/ISCSI}{Als Blockdevice}
\end{itemize}

\section{Zielvorstellung}
Aus der Idee ergibt sich dann, grob skizziert, folgende Zielvorstellung:

\jpaimg{./images/ZBoxDS411Slim.png}{Zielvorstellung}

\section{Auswahl des Host-OS}
Das Projektes beginnt am 31. Dezember 2018. Grundsätzlich könnte jede Linux Variante als Host Betriebssystem gewählt
werden. Aber Linux wäre ja nicht Linux, wenn es nur eine Distribution für das Hosten von Docker Containern gäbe. Zur
Auswahl stehen unter anderem:
\begin{itemize}
	\item \href{https://alpinelinux.org/}{Alpine Linux}
	\item \href{https://www.openmediavault.org/}{OpenMediaVault}
	\item \href{https://vmware.github.io/photon/}{Photon}
	\item \href{https://wiki.smartos.org/display/DOC/Welcome+to+SmartOS}{SmartOS}
\end{itemize}

Von jedem OS wurde ein Testsystem in einer \href{https://www.virtualbox.org/}{Virtual Box} installiert und kurz
angetestet. Wichtig ist für mich, daß keine speziellen Befehle zur Verwendung kommen, sondern das sich das System sowie
das Handling an die von mir gewohnten \href{https://www.debian.org/index.de.html} {Debian} basierten Systemen anlehnt.

Nach verschiedenen Tests diesbezüglich habe ich mich erst einmal für Alpine Linux entschieden. Nicht nur für Container
selbst wird das gerne verwendet; durch seine geringe Anforderungen kann es auch gut als Host-OS eingesetzt werden.

\section{Erstinstallation}
\subsection{Das Host-OS}
Zur Installation habe ich das \href{http://dl-cdn.alpinelinux.org/alpine/v3.8/releases/x86_64/alpine-standard-3.10.2-x86_64.iso}{Standard Image}\footnote{Zum
Zeitpunkt der Veröffentlichung} verwendet. Davon ist schnell gebootet. Die\linebreak Anmeldung mit dem Benutzer \code{root}
verlangt kein Kennwort. Den eigentlichen Installationsprozess startet man mit \code{setup-alpine}. Bei den abgefragten
Einstellungen benutze ich im Regelfall den Default-Wert. Abweichend davon sind

\begin{itemize}
    \item Die Tastatur mit Layout \code{de} und Variante  \code{de}
    \item Die Zeitzone mit \code{Europe/Berlin}
    \item Die Festplatte \code{sda} als \code{sys}
\end{itemize}

Das System ist dann in weniger als 10 Minuten aufgesetzt. Damit der Server auch immer die gleiche IP bekommt, richte ich
das in meinem Router auch entsprechend ein. Damit kann ich innerhalb des Netzwerks immer auf die gleiche IP
zurückgreifen. Alternativ könnte man auch eine statische IP auf dem Server selbst einrichten. Allerdings habe ich das
bei keinem meiner Geräte so gemacht, sondern alle auf meinem Router entsprechend eingetragen.

\subsection{Vorbereitungen}
Damit ich nicht mit dem \code{root} Benutzer arbeiten muss, lege ich einen Benutzer für mich an. Davor installiere ich
noch das Programm \href{https://de.wikipedia.org/wiki/Sudo}{sudo}, damit der neue Benutzer Root Berechtigung bekommen
kann. Ergänzend kommt noch \href{https://curl.haxx.se/}{curl} hinzu, um über die Kommandozeile einfache Downloads zu
ermöglichen. Ausserdem wird mit \href{https://hisham.hm/htop/index.php?page=main}{htop} noch ein übersichtlicherer
Prozessmanager installiert.

\begin{figure}[H]
\begin{lstlisting}
# Install some tools which are default by other linux
apk add sudo htop curl
adduser admin
visudo
# Add the line at the end
# admin    ALL=(ALL) ALL
\end{lstlisting}
\caption{Host Benutzer admin}\label{fig:Host Benutzer admin}
\end{figure}

Damit kann ich mich mit \href{https://www.chiark.greenend.org.uk/~sgtatham/putty/latest.html}{Putty} remote an dem
System anmelden und trotzdem alles mit \code{sudo} und Root Berechtigungen ausführen. In einem weiteren Schritt schalte
ich \href{https://de.wikipedia.org/wiki/IPv6}{IPv6} ab. Die Erläuterung dazu folgt später.

\begin{figure}[H]
\begin{lstlisting}
vi /etc/sysctl.conf
# Add following line
net.ipv6.conf.all.disable_ipv6 = 1
\end{lstlisting}
\caption{Host IPv6 abschalten}\label{fig:IPv6 abschalten}
\end{figure}

\section{Die Verbindung zum NAS}
Bei meinen Versuchen mit verschiedenen Konstellationen von \href{https://de.wikipedia.org/wiki/Network_File_System}{NFS}-
und \href{https://de.wikipedia.org/wiki/Server_Message_Block}{SMB}-Shares habe ich herausgefunden, dass das alles mehr
oder weniger suboptimal ist.

Docker selbst legt beim Start von Containern verschiedene mounts an. Damit gab es dann Konflikte mit Zugriffsrechten bei
der Verarbeitung von Dateien. So funktionierte zwar \href{https://hub.docker.com/r/portainer/portainer/}{Portainer} ohne
Probleme, aber schon \href{https://hub.docker.com/r/sdorra/scm-manager}{sdorra/scm-manager} verweigerte den Start wegen
einem Berechtigungsproblem. Auch \href{https://hub.docker.com/r/yobasystems/alpine-mariadb}{yobasystems/alpine-mariadb}
konnte ich nicht ans Laufen bringen. Und das völlig unabhängig davon, ob

\begin{itemize}
    \item Ein Share unter \code{/var/lib/docker/volume} gemountet wurde,
    \item Ein Share unter \code{/mnt/nasshare} gemountet wurde,
    \item Als Share \href{https://de.wikipedia.org/wiki/Network_File_System}{NFS} verwendet wurde,
    \item Als Share \href{https://de.wikipedia.org/wiki/Server_Message_Block}{SMB} verwendet wurde,
    \item Ein bzw.\ mehrere absolute Pfade angegeben wurden,
    \item Ein bzw.\ mehrere \href{https://nerdblog.steinkopf.net/2017/12/persistente-docker-volumes-mit-nfs-und-cifs/}{Volumes mit NFS} verwendet wurden,
    \item Ein bzw.\ mehrere \href{https://nerdblog.steinkopf.net/2017/12/persistente-docker-volumes-mit-nfs-und-cifs/}{Volumes mit SMB} verwendet wurden.
\end{itemize}

Das Ergebnis war niederschmetternd.

Erst als ich einen Versuch mit \href{https://de.wikipedia.org/wiki/ISCSI}{iSCSI} gemacht habe, waren meine Anstrengungen
von Erfolg gekrönt. Mit iSCSI wird über das Netzwerk ein Block\-device zur Verfügung gestellt. Beim Start des HostOS
wird das Blockdevice dann wie eine konventionelle Festplatte in das System eingebunden. Docker merkt dann nicht, dass
die Festplatte irgendwo im Netzwerk liegt und verwendet sie wie eine lokale Festplatte. Damit haben dann oben genannte
Container funktioniert.

\subsection{iSCSI-Target auf dem NAS anlegen}
Damit auf der ZBox die virtuelle Festplatte eingebunden werden kann, muss sie vorher auf dem NAS angelegt werden. Das
geht mit dem iSCSI-Manager.

\jpaimg{./images/DSM-ISCSI-01.png}{Der initiale Start des iSCSI-Managers}

\jpaimg{./images/DSM-ISCSI-02.png}{Die iSCSI-Target Liste}

\jpaimg{./images/DSM-ISCSI-03.png}{Das iSCSI-Target anlegen}

\jpaimg{./images/DSM-ISCSI-04.png}{Das dazugehörige LUN mit 100GB anlegen}

Die Größe von 100 GB ist hier exemplarisch zu sehen. Je nachdem, wieviel man von der NAS-Kapazität nutzen möchte, muss
das natürlich angepasst werden.

\jpaimg{./images/DSM-ISCSI-05.png}{Die Zusammenfassung}

\jpaimg{./images/DSM-ISCSI-06.png}{Das Ergebnis}

Wichtig ist dann noch, dass prinzipiell nur ein Netzwerkprotokoll aktiv ist. Ansonsten versucht die ZBox, sich auf jede
IP zu verbinden. Da das Host OS sowohl \href{https://de.wikipedia.org/wiki/IPv4}{IPv4} als auch \href{https://de.wikipedia.org/wiki/IPv6}{IPv6}
verwendet, kann das zu Problemen führen. Ich habe mich dazu entschieden, auf dem NAS ausschließlich \href{https://de.wikipedia.org/wiki/IPv4}{IPv4}
zu verwenden.

\subsection{iSCSI auf der ZBox}
Nachdem das iSCSI-Target auf dem NAS angelegt wurde, wird es nun auf der ZBox eingebunden. Das Ergebnis ist eine Mischung
aus der \href{https://wiki.alpinelinux.org/wiki/ISCSI_Raid_and_Clustered_File_Systems}{Anleitung für eine Initiator Config}
sowie diesem \href{https://kifarunix.com/how-to-install-and-configure-iscsi-storage-server-on-ubuntu-18-04/}{How-To}.

\begin{figure}[H]
\begin{lstlisting}
# Install required packages
sudo apk add open-iscsi util-linux

# Modify config to startup automatically
sudo vi /etc/iscsi/iscsid.conf
# Set node.startup = automatic

# Set initiator name
sudo vi /etc/iscsi/initiatorname.iscsi
# Replace the existing line with the name of the target provided by the NAS
# InitiatorName=iqn.2000-01.com.synology:DS411Slim.Target

# Add service to default runlevel and start it
sudo rc-update add iscsid default
sudo rc-service iscsid start
\end{lstlisting}
\caption{Installation von open-iscsi}\label{fig:Installation von open-iscsi}
\end{figure}

\begin{figure}[H]
\begin{lstlisting}
# Discover possible targets
sudo iscsiadm -m discovery -t sendtargets -p <IP OF NAS>
# Login to specific target
sudo iscsiadm -m node -T <TARGET NAME OF NAS> -l
\end{lstlisting}
\caption{Einbinden des iSCSI-Target}\label{fig:Einbinden des iSCSI-Target}
\end{figure}

\begin{figure}[H]
\begin{lstlisting}
# Figure out new attached drive
sudo lsblk --scsi
# NAME HCTL       TYPE VENDOR   MODEL             REV TRAN
# sda  0:0:0:0    disk ATA      VBOX HARDDISK    1.0  sata
# sdb  3:0:0:1    disk SYNOLOGY iSCSI Storage    3.1  iscsi
# sr0  2:0:0:0    rom  VBOX     CD-ROM           1.0  ata
# NAS is connected as /dev/sdb

sudo fdisk /dev/sdb
# n - Add new partition
# p - Primary partition
# 1 - Partition number
# <enter> - First sector
# <enter> - Last sector
# w - Write partition table

# Format partition 1
sudo mkfs.ext4 /dev/sdb1
\end{lstlisting}
\caption{Filesystem einrichten}\label{fig:Filesystem einrichten}
\end{figure}

\begin{figure}[H]
\begin{lstlisting}
# Create mount point
sudo mkdir /var/lib/docker

# Figure out UUID of /dev/sdb1
sudo blkid /dev/sdb1

# Write entry to /etc/fstab
sudo vi /etc/fstab

# Add following line
# UUID=<UUID OF /dev/sdb1>   /var/lib/docker   ext4   _netdev   0 0

# Start netmount on boot
sudo rc-update add netmount

# Reboot the system
sudo reboot
\end{lstlisting}
\caption{Automount einrichten}\label{fig:Automount einrichten}
\end{figure}

\subsection{Nach dem Reboot}
Nach einem Reboot sollte dann das Blockdevice an dem als \code{/dev/sdb1} sichtbar sein.
Die Festplatte wurde unter \code{/var/lib/docker} eingebunden. Das sieht in etwa so aus:

\begin{figure}[H]
\begin{lstlisting}
$ df
Filesystem           1K-blocks      Used Available Use% Mounted on
devtmpfs                 10240         0     10240   0% /dev
shm                    2023296         0   2023296   0% /dev/shm
/dev/sda3             16346492    724524  14771896   5% /
tmpfs                   404660       116    404544   0% /run
/dev/sda1                95054     20749     67137  24% /boot
/dev/sdb1            102687640     61464  97366916   0% /var/lib/docker
\end{lstlisting}
\caption{Anzeige der Laufwerke}\label{fig:Anzeige der Laufwerke}
\end{figure}

\section{Docker Umgebung}
Mit Docker wird die Umgebung installiert, um mit Containern arbeiten zu können. Dazu zählt nicht nur Docker selbst,
sondern auch Docker Compose.

\subsection{Installation von Docker}
Docker ist schnell installiert -- gibt es doch einen guten \href{https://wiki.alpinelinux.org/wiki/Docker}{Artikel}
hierzu in dem Wiki von Alpine Linux. Zuerst muss man in \code{/etc/apk/repositories} noch die
\href{http://dl-cdn.alpinelinux.org/alpine/latest-stable/community}{Quelle} für Docker eintragen, dann ein \code{apk update}
ausführen und schon kann man mit \code{apk add docker} Docker installieren.

Damit Docker auch bei einem Neustart gestartet wird, muss man das noch dem System mit \code{rc-update add docker default}
mitteilen.

\begin{figure}[H]
\begin{lstlisting}
sudo vi /etc/apk/repositories
# Add the following row without the hash:
# http://dl-cdn.alpinelinux.org/alpine/edge/community
sudo apk update
sudo apk add docker
sudo rc-update add docker default
sudo service docker start
\end{lstlisting}
\caption{Installation von Docker}\label{fig:Installation von Docker}
\end{figure}

\subsection{Installation von Docker Compose}
Um später auch mit \href{https://docs.docker.com/compose/}{Docker Compose} arbeiten zu können, installieren wir das
gleich mit. Dazu folgen wir der offiziellen \href{https://docs.docker.com/compose/install/}{Anleitung}.

\begin{figure}[H]
\begin{lstlisting}
sudo apk add py-pip python-dev libffi-dev openssl-dev gcc libc-dev make
sudo pip install --upgrade pip
sudo pip install docker-compose
\end{lstlisting}
\caption{Installation von Docker Compose}\label{fig:Installation von Docker Compose}
\end{figure}

\subsection{Abhängigkeiten}
Da sowohl Docker als auch das Netzlaufwerk im Runlevel \code{default} gestartet werden, kann es unter Umständen zu einem
Problem kommen: Docker wird gestartet, bevor das Netz\-laufwerk verbunden ist. Das wäre sehr unangenehm, denn dann würde
kein Container starten. Also sorgen wir dafür, daß das nicht zu einem Problem wird, indem wir Docker noch die
Abhängigkeiten von \code{iscsid}, dem Daemon für das iSCSI, sowie \code{netmount}, dem Verbinden von Netzlaufwerken,
mitteilen.

\begin{figure}[H]
\begin{lstlisting}
sudo vi /etc/init.d/docker
# Extend
#
# depend() {
#        need sysfs cgroups
# }
#
# With
#
# depend() {
#        need sysfs cgroups iscsid netmount
# }
\end{lstlisting}
\caption{Docker Abhängigkeiten}\label{fig:Docker Abhängigkeiten}
\end{figure}

\section{Die Container}
Hier wird die Installation von verschiedenen Containern beschrieben, welche ich auf meinem Server verwenden möchte.

\subsection{Containerliste}
Die Container sollen folgende Dienste zur Verfügung stellen -- aufgeführt in der Reihenfolge, wie sie installiert werden:

\begin{figure}[H]
\begin{itemize}
\item \href{https://www.portainer.io/}{Portainer} -- Oberfläche für Docker.
\item \href{https://www.scm-manager.org/}{SCM-Manager} -- Oberfläche zu den Versionierungssystememen \href{https://git-scm.com/}{Git}
und \href{https://subversion.apache.org/}{Subversion}.
\item \href{https://mariadb.org/}{MariaDB} -- Datenbank für diverse Softwareprojekte.
\item \href{https://www.phpmyadmin.net/}{phpMyAdmin} -- Oberfläche für die Datenbank.
\item \href{https://mosquitto.org/}{Mosquitto} -- Ein \href{https://mqtt.org/}{MQTT} Broker, ebenfalls für diverse
Softwareprojekte.
\item \href{https://syncthing.net/}{Syncthing} -- Filesynchronisation zwischen PC und dem Server.
\item \href{https://nextcloud.com/}{Nextcloud} -- Eine lokale Cloud.

\item \href{https://www.samba.org/}{Samba} -- Dateiserver.
\item \href{https://www.clamav.net/}{ClamAV} -- (Anti-) Viren Scanner.
\item \href{https://restic.net/}{restic} -- Datensicherung.
\item \href{https://docs.min.io/}{MinIO} -- Als Ziel für die Datensicherung.
\item \href{https://pi-hole.net/}{Pi-Hole} -- Werbeblocker für das lokale Netzwerk.
\end{itemize}
\caption{Containerliste}\label{fig:Containerliste}
\end{figure}

Nicht aufgezählt sind dann die Container, die in der Zukunft noch dazu kommen werden -- was auch immer mir da noch
einfallen mag\ldots

\subsection{Portainer}\label{sub:Portainer}
Damit man sich nicht so mit der Kommandozeile herumschlagen muß, wird \href{https://hub.docker.com/r/portainer/portainer/}{Portainer}
als Frontend zu Docker installiert. Damit sollte es etwas komfortabler sein, die gewünschten Ziele zu erreichen. Damit
die Daten/Einstellungen persistent gemacht werden können, braucht es ein Docker Volume.

\begin{figure}[H]
\begin{lstlisting}
sudo docker volume create --name v_portainer
sudo docker run -dit \
    --restart unless-stopped \
    --name portainer \
    -p 9000:9000 \
    -v /var/run/docker.sock:/var/run/docker.sock \
    -v v_portainer:/data \
    -d portainer/portainer:latest
\end{lstlisting}
\caption{Installation von Portainer}\label{fig:Installation von Portainer}
\end{figure}

Wichtig ist hierbei der Teil \code{--restart unless-stopped} -- damit wird der Docker Container automatisch beim
Hochfahren des Systems gestartet. Portainer selbst ist eine Webapplikation, die man über den entsprechenden Port
(Default ist 9000) aufrufen kann. Bei der ersten Anmeldung muss man einen Account anlegen und ein Passwort dazu vergeben.

\jpaimg{./images/Portainer-01.png}{Erste Anmeldung an Portainer}

Im zweiten Schritt gibt man an, ob man mit Portainer den lokalen Host oder einen Remote Host verwalten möchte. Hier ist
ganz klar der lokale Host gewünscht.

\jpaimg{./images/Portainer-02.png}{Verbindung von Portainer an Localhost}

Nachdem die \jpaquote{Formalitäten} erledigt sind, kann man Portainer verwenden. Die Startseite sieht für unser System
erst einmal so aus:

\jpaimg{./images/Portainer-03.png}{Die Portainer Startseite}

\subsection{Versionierungssystem}\label{sub:SCM-Manager}
Als \href{https://de.wikipedia.org/wiki/Versionsverwaltung}{Versionsverwaltung} möchte ich ein Subversion einsetzen.
Und zwar mit einer Web\-oberfläche. \href{https://www.scm-manager.org/}{SCM-Manager} scheint hierfür ganz gut geeignet
zu sein. Damit habe ich später die Möglichkeit, von Subversion auf Git zu wechseln.

\subsubsection{Installation des Docker Containers}
Verwendet wird der offizielle \href{https://hub.docker.com/r/sdorra/scm-manager}{Docker Container} des SCM-Managers.

\begin{figure}[H]
\begin{lstlisting}
sudo docker volume create --name v_scm_manager
sudo docker run -dit \
    --restart unless-stopped \
    --name scm_manager \
    -p 8080:8080 \
    -v v_scm_manager:/var/lib/scm \
    -d sdorra/scm-manager:latest
\end{lstlisting}
\caption{Installation der Versionsverwaltung}\label{fig:Installation der Versionsverwaltung}
\end{figure}

\subsubsection{Initiale Konfiguration}
Nachdem das Volume angelegt ist und der Container gestartet wurde, kann man über \code{http://<IP OF SERVER>:8080} die
Weboberfläche aufrufen. Der Standard Login für den SCM-Manager lautet \code{scmadmin}/\code{scmadmin}. Ein Repository
ist schnell aus einem Dump importiert, dazu noch ein Benutzer angelegt und schon kann man über die URL\linebreak
\code{http://<USER>@<IP OF SERVER>:8080/scm/svn/<REPOSITORY>/} das Repo auschecken\linebreak und bearbeiten.

\subsubsection{SCM-Manager Plugins}
Der SCM-Manager läßt sich mit Plugins erweitern. Beispiele für solche Plugins sind:

\begin{itemize}
\item \href{https://bitbucket.org/sdorra/scm-activity-plugin/src/default/}{SCM Activity} zur Anzeige der letzten Aktivitäten im Repository.
\item \href{https://bitbucket.org/sdorra/scm-statistic-plugin/src/default/}{Statistics} zeigt statistische Informationen zum Repository.
\end{itemize}

\subsection{MariaDB}\label{sub:MariaDB}
Für verschiedene Dinge wird eine Datenbank benötigt. Der Docker Container, den ich ausgewählt habe, ist
\href{https://hub.docker.com/r/yobasystems/alpine-mariadb}{yobasystems/alpine-mariadb}, weil damit wirklich einfach eine
neue DB-Instanz aufgesetzt werden kann. Das geht ganz schnell, und zwar so:

\begin{figure}[H]
\begin{lstlisting}
sudo docker volume create --name v_mariadb
sudo docker run -dit \
    --restart unless-stopped \
    --name mariadb \
    -p 3306:3306 \
    -v v_mariadb:/var/lib/mysql \
    -e MYSQL_ROOT_PASSWORD=<SECRET PASSWORD HERE> \
    -d yobasystems/alpine-mariadb:latest \
    --character-set-server=utf8mb4 \
    --collation-server=utf8mb4_unicode_ci
\end{lstlisting}
\caption{Installation der MariaDB}\label{fig:Installation der MariaDB}
\end{figure}

Damit wird das Datenbanksystem mit einem \href{https://de.wikipedia.org/wiki/UTF-8}{UTF-8} Zeichensatz initialisiert und
das \code{Root-Passwort} wird gesetzt. Alles weitere kann man über die Oberfläche \nameref{sub:phpMyAdmin}
einstellen/erledigen.

\subsection{phpMyAdmin}\label{sub:phpMyAdmin}
Zur Administration der MariaDB gibt es quasi das Tool schlechthin: \href{https://www.phpmyadmin.net/}{phpMyAdmin}. Und
es gibt einen offiziellen \href{https://hub.docker.com/r/phpmyadmin/phpmyadmin}{Docker Container} dazu. Diesen möchte
ich verwenden, um die Datenbank damit zu administrieren.

\begin{figure}[H]
\begin{lstlisting}
sudo docker run -dit \
    --name phpmyadmin \
    -p 8888:80 \
    --link mariadb:db \
    -d phpmyadmin/phpmyadmin:latest
\end{lstlisting}
\caption{Installation von phpMyAdmin}\label{fig:Installation von phpMyAdmin}
\end{figure}

Als Port wurde bewusst \code{8888} verwendet, da der Port \code{8080} bereits vom \nameref{sub:SCM-Manager} verwendet
wird. Zudem wurde darauf verzichtet, den Container immer gestartet zu haben. Dieser soll nur bei Bedarf gestartet und
danach wieder beendet werden. Deswegen fehlt das Argument \code{--restart unless-stopped} im Aufruf. Die Anmeldung am
UI erfolgt mit Benutername/Passwort \code{root}/\code{<ROOT PASSWORT>} -- dem Passwort, welches bei der Erstellung des
\nameref{sub:MariaDB} Containers angegeben wurde.

\subsection{Mosquitto}
Da das \href{https://de.wikipedia.org/wiki/Internet_der_Dinge#targetText=Das%20Internet%20der%20Dinge%20(IdD,und%20Kommunikationstechniken%20zusammenarbeiten%20zu%20lassen.}{IoT},
also das Internet der Dinge, immer mehr an Bedeutung gewinnt, habe ich mich für diverse Softwareprojekte auch mit diesem
Thema auseinander gesetzt. Dazu gehört unter anderem das Protokoll \href{https://mqtt.org/}{MQTT}.
\href{https://mosquitto.org/}{Mosquitto} ist ein Broker, quasi das Herzstück dazu. Verwendet wird ist das offizielle
\href{https://hub.docker.com/_/eclipse-mosquitto}{eclipse-mosquitto} Image.

\begin{figure}[H]
\begin{lstlisting}
sudo docker volume create --name v_mosquitto_config
sudo docker volume create --name v_mosquitto_data
sudo docker volume create --name v_mosquitto_log
sudo docker run -dit \
    --restart unless-stopped \
    --name mosquitto \
    -p 1883:1883 \
    -p 9001:9001 \
    -v v_mosquitto_config:/mosquitto/config \
    -v v_mosquitto_data:/mosquitto/data \
    -v v_mosquitto_log:/mosquitto/log \
    -d eclipse-mosquitto:latest
\end{lstlisting}
\caption{Installation von Mosquitto}\label{fig:Installation von Mosquitto}
\end{figure}

Damit ist der MQTT Broker einsatzbereit und kann beispielsweise mit dem \href{https://mqtt-explorer.com/}{MQTT Explorer}
getestet werden.

\subsection{Syncthing}
Das Synology NAS bringt von Haus aus eine Software mit, über die lokale Ordner und Dateien auf das NAS synchronisiert
werden können. Da das NAS aber nur noch als Fileserver dienen soll, muss dazu eine Alternative verwendet werden.
\href{https://syncthing.net/}{Syncthing} scheint mir dazu gut geeignet zu sein. Vor allem, da man diese Software nicht
nur auf dem PC, sondern auch auf dem Smartphone laufen lassen kann.

\begin{figure}[H]
\begin{lstlisting}
sudo docker volume create --name v_syncthing
sudo docker run -dit \
    --restart unless-stopped \
    --name syncthing \
    -p 8384:8384 \
    -p 22000:22000 \
    -v v_syncthing:/var/syncthing \
    -d syncthing/syncthing:latest
\end{lstlisting}
\caption{Installation von Syncthing}\label{fig:Installation von Syncthing}
\end{figure}

Damit ist Syncthing initial eingerichtet. Die Startseite \code{http:\textbackslash\textbackslash<IP OF SERVER>:8384} von
Syncthing meldet sich dann so:

\jpaimg{./images/Syncthing-01.png}{Syncthing, erster Aufruf}

Es wird empfohlen, für die Oberfläche ein Kennwort zu vergeben. Das geht über das Menü oben rechts, \code{Aktionen},
\code{Einstellungen} und dort den Reiter \code{GUI}:

\jpaimg{./images/Syncthing-02.png}{Syncthing, GUI Kennwort}

Danach meldet sich Synthing mit der Standard Startseite. Die weitere Konfiguration bleibt dem Benutzer überlassen.

\jpaimg{./images/Syncthing-03.png}{Syncthing, Standard Startseite}

Als Client für Windows Systeme wird auf der \href{https://syncthing.net/}{Syncthing Seite} das Programm \href{https://github.com/canton7/SyncTrayzor}{SyncTrayzor}
empfohlen.

\subsection{NextCloud}
Die lokale Cloud basiert auf \href{https://nextcloud.com/}{NextCloud}. Von NextCloud wird ein offizielles Image zur
Verfügung gestellt. Hier möchte ich die bereits installierte Version der Datenbank MariaDB nutzen und dort eine
Datenbank für NextCloud zur Verfügung stellen. Deswegen lege ich dort eine Datenbank \code{nextcloud} an. Damit die
Cloud auch Zugriff darauf hat, wird der Benutzer \code{nextcloud} mit dem Kennwort \code{<SECRET PASSWORD HERE>}
verwendet.

\begin{figure}[H]
\begin{lstlisting}
sudo docker volume create --name v_nextcloud
sudo docker run -dit \
    --restart unless-stopped \
    --name nextcloud \
    -e MYSQL_DATABASE=nextcloud \
    -e MYSQL_USER=nextcloud \
    -e MYSQL_PASSWORD=<SECRET PASSWORD HERE> \
    -e MYSQL_HOST=<IP OF SERVER>:3306 \
    -p 8765:80 \
    -v v_nextcloud:/var/www/html \
    -d nextcloud:latest
\end{lstlisting}
\caption{Installation von NextCloud}\label{fig:Installation von NextCloud}
\end{figure}

Wenn man das \nameref{sec:AIOFile} verwendet, sollte man vor dem ersten Aufruf von NextCloud natürlich über
\nameref{sub:phpMyAdmin} den Benutzer und die Datenbank angelegt haben.

\subsection{Dateiserver}
Mit dem Dateiserver habe ich mich lange Zeit schwer getan. Das ist der eigentliche Grund, warum das Projekt so viel
Zeit in Anspruch nimmt. Denn normalerweise legen die Docker Container ihre Dateien lokal ab. Meine Datenmenge, also
meine Filme, Musik und sonstige Dateien passen nicht lokal auf die SSD der ZBox. Die müssen also nach wie vor auf dem
NAS liegen. Durch Experimente habe ich dann den für mich besten Weg gefunden: Die Einbindung mit iSCSI und den Rest
mit Docker.

Bisher habe ich verschiedene Features über Docker abgebildet, welche auf meinem NAS laufen. Dazu zählt der
\nameref{sub:SCM-Manager}, die \nameref{sub:MariaDB}, die Administrationsoberfläche \nameref{sub:phpMyAdmin} dazu,
ebenso wie die Administrationsoberfläche für Docker, \nameref{sub:Portainer}.

Den eigentlichen Dateiserver habe ich bis jetzt noch nicht angepackt. Da auf meine NAS bereits ein
\href{https://www.samba.org/}{Samba} lief, bin ich davon ausgegangen, einen Container zu schnappen, die Konfiguration
zu übertragen und dann klappt alles. Das war mal wieder ein typischer Fall von \jpaquote{Denkste!} -- so einfach ist das
leider nicht.

Das liegt zum einen daran, dass Synology eine eigene Version von Samba einsetzt. Damit sind Teile der Konfigurationsdatei
von dem NAS nicht mit einem offiziellen Samba zu verwenden. Zum anderen liegt das aber auch daran, dass Samba eben keine
einfache Applikation ist. Somit wird ein flexibler Container benötigt, der auch die Anforderungen erfüllt.

\subsubsection{Samba Image}
Zum Glück hat \href{https://github.com/dperson}{David Personette} einen solchen \href{https://hub.docker.com/r/dperson/samba}{Container}
bereits erstellt. Vielen Dank an dieser Stelle. Diesen Container werde ich dann für mein Projekt verwenden.
Grundsätzlich wird das eine mehrteilige Aufgabe werden. Zum einen werde ich den Container in meinen Stack integrieren.
Zum anderen werde ich verschiedene Shell Scripte erstellen, um Benutzer, Shares und anderes in dem Container zu
konfigurieren. Der Container lässt sich mit dem Image von David Personette ganz einfach einrichten:

\begin{figure}[H]
\begin{lstlisting}
sudo docker volume create --name v_samba_cache
sudo docker volume create --name v_samba_data
sudo docker volume create --name v_samba_etc
sudo docker volume create --name v_samba_lib
sudo docker volume create --name v_samba_log
sudo docker volume create --name v_samba_run
sudo docker run -dit \
    --name samba \
    -p 139:139 \
    -p 445:445 \
    -e TZ=DE \
    -m 512m \
    -v v_samba_cache:/var/cache/samba \
    -v v_samba_data:/mount \
    -v v_samba_etc:/etc \
    -v v_samba_lib:/var/lib/samba \
    -v v_samba_log:/var/log/samba \
    -v v_samba_run:/run/samba \
    -d dperson/samba:latest
sudo docker exec -i -t samba /usr/bin/samba.sh -w "WLP"
sudo docker exec -i -t samba /usr/bin/samba.sh -g "log level = 2"
sudo docker exec -i -t samba /usr/bin/samba.sh -u "<USER>;<SECRET PASSWORD HERE>"
sudo docker exec -i -t samba /usr/bin/samba.sh -s "public;/share"
sudo docker container restart samba
\end{lstlisting}
\caption{Samba Container}\label{fig:Samba Container}
\end{figure}

\subsection{ClamAV}

\subsection{restic}

\subsection{MinIO}

\subsection{Pi-Hole}

\section{All in One Docker File}\label{sec:AIOFile}
Die Container habe ich zuerst manuell eingerichtet, um die für mich relevanten Einstellungen herauszufinden. Aber bei
meinen vielen Versuchen in einer \href{https://www.virtualbox.org/}{Virtual Box} wurde mir das auf die Dauer zu lästig.
Zudem wollte ich auf meiner ZBox nicht ebenfalls alles manuell einrichten. Deswegen habe ich die einzelnen Schritte bzw.
Container in einer Docker Compose Datei zusammengefasst. Die sieht dann in etwa so aus:
\\
\lstinputlisting[caption={All in One Setup},label={All in One Setup},captionpos=b]{./server/aio/docker-compose.yml}

\section{FAQ}
\begin{tabularx}{\textwidth}{@{}l @{\hspace{0.1cm}}X}
\textbf{Q:} & Wie bekomme ich eine Shell für bzw.\ in einen Container? \\
\textbf{A:} & Dazu gibt man in der Kommandozeile folgendes ein: \\
\end{tabularx}

\begin{figure}[H]
\begin{lstlisting}
# Maybe there is no bash available, then try with /bin/sh
sudo docker exec -i -t <NAME OF CONTAINER> /bin/bash
\end{lstlisting}
\caption{Container Shell}\label{fig:Container Shell}
\end{figure}

\begin{tabularx}{\textwidth}{@{}l @{\hspace{0.1cm}}X}
\textbf{Q:} & Wie kann ich das MariaDB root-Passwort ändern? \\
\textbf{A:} & Dazu loggt man sich mit dem root-Account und dem bekannten Passwort ein. Danach führt man folgende SQL
Statements aus: \\
\end{tabularx}

\begin{figure}[H]
\begin{lstlisting}
UPDATE mysql.user SET Password=PASSWORD('<NEW PASSWORD HERE>') WHERE User='root';
FLUSH PRIVILEGES;
\end{lstlisting}
\caption{MariaDB Root Password ändern}\label{fig:MariaDB Root Password ändern}
\end{figure}

\begin{tabularx}{\textwidth}{@{}l @{\hspace{0.1cm}}X}
\textbf{Q:} & Was ist ein Docker Stack? \\
\textbf{A:} & Als einen Docker Stack bezeichnet man einen oder mehrere Container, die mit weiteren Einstellungen in
einer \code{docker-compose.yml} Datei zusammengefasst sind. \\
\end{tabularx}

\begin{tabularx}{\textwidth}{@{}l @{\hspace{0.1cm}}X}
\textbf{Q:} & Wie kann ich einen Docker Stack starten/anhalten? \\
\textbf{A:} & Dazu führt man auf der Kommandozeile folgendes aus: \\
\end{tabularx}

\begin{figure}[H]
\begin{lstlisting}
cd <NAME OF STACK>
# To start the docker stack
# The option '-d' runs the stack in the background
sudo docker-compose up -d
# To stop the docker stack
sudo docker-compose stop
\end{lstlisting}
\caption{Start/Stop Docker Stack}\label{fig:Start/Stop Docker Stack}
\end{figure}

\begin{tabularx}{\textwidth}{@{}l @{\hspace{0.1cm}}X}
\textbf{Q:} & Wie kann ich eine Datei aus einem Docker Container kopieren? \\
\textbf{A:} & Dazu führt man auf der Kommandozeile folgendes aus: \\
\end{tabularx}

\begin{figure}[H]
\begin{lstlisting}
sudo docker cp <NAME OF CONTAINER>:<FILE PATH WITHIN CONTAINER> <HOST PATH TARGET>
\end{lstlisting}
\caption{Datei aus Container kopieren}\label{fig:Datei aus Container kopieren}
\end{figure}

\newpage
\phantomsection{}
\addcontentsline{toc}{section}{Abbildungsverzeichnis}
\listoffigures\thispagestyle{fancy}
\end{document}
